\subsection{Part 1: Sessions of the Students'
Senate}\label{part-1-sessions-of-the-students-senate}

\begin{enumerate}
\def\labelenumi{\arabic{enumi}.}
\item
  The membership, duties, powers and privileges of the Students' Senate
  are defined in the Constitution of the Students' Gymkhana.
\item
  The notification for sessions of the Students' Senate shall be sent
  out along with the agenda before the scheduled start of the session
  to:

  \begin{enumerate}
  \def\labelenumii{\alph{enumii}.}
  \item
    All Members of the Students' Senate
  \item
    The General Body of the Students
  \item
    The Advisory Body, Students' Gymkhana
  \end{enumerate}
\item
  The agenda for the session of the Students' Senate shall be finalized
  by the Chairperson, Students' Senate with the help of the Steering
  Committee of the Students' Senate and made available with the
  supporting documents in the public domain before the scheduled start
  of the session.
\item
  The agenda of the session and the order of business may be overruled
  by a simple majority in the Students' Senate. Ordinarily, a call for
  agenda items from the General Body shall be made at least four days,
  before the scheduled start of the session and the final agenda of the
  session shall be made available at least 24 hours prior to the
  scheduled start of the session The sessions of the Students' Senate
  for which the above process of agenda preparation is not followed or
  are requisitioned under Article 7.09 of the Constitution or its
  Appendices shall be known as special sessions. The order of business
  shall be given by the Steering Committee. The agenda of such a session
  may not be overturned by the Students' Senate under A.1.3 without the
  consent of the Chairperson and the requisitioning authority.
\item
  The sessions of the Students' Senate convened to discuss matters of
  emergency shall be known as emergency sessions. These may be convened
  by the Chairperson on his/her own accord or as requisitioned under any
  relevant clause of the Constitution or these rules and procedures.
  Such sessions shall have reduced quorum and notice requirements under
  Clause A.1.9 of these rules and procedures.
\item
  The Students' Senate shall convene for a regular session at least four
  times during a regular semester.
\item
  The agenda for Regular sessions shall be prepared on the basis of a
  Call for Agenda made to the General Body at least 4 days before the
  scheduled sessions. The Order of Business during Regular sessions of
  the Students' Senate shall be as follows:

  \begin{enumerate}
  \def\labelenumii{\alph{enumii}.}
  \item
    Ratification of draft minutes
  \item
    Report on the actions taken on the decisions of the Senate
  \item
    Announcements by the Chairperson and Executives
  \item
    Questions and Remarks allowed by the Chairperson
  \item
    Reports by all Executive Bodies
  \item
    Reports of Sub-Committees
  \item
    Unfinished business
  \item
    New business
  \item
    Questions and Remarks
  \end{enumerate}
\item
  One-half of the total strength of the Students' Senate shall
  constitute quorum.The Students' Senate shall convene only after quorum
  has been achieved. If quorum is absent at any point during a session,
  the Chairperson shall adjourn the session for lack of quorum without
  transacting any further business. Sessions adjourned for lack of
  quorum shall require quorum for reconvening.
\item
  For emergency sessions, one-fourth of the total strength of the
  Students' Senate along with the presence of at least one Senator from
  every batch with more than 4 elected Senators shall constitute quorum.
  The decisions taken in the emergency quorum shall be taken up for
  ratification by the full Students' Senate in the next full session.
  The decision taken shall be enforced until the ratification.
\item
  For a session requisitioned under the provision of Article 7.10 of the
  Constitution, adjourned once due to lack of quorum, the Chairperson
  will call another session within 48 hours. For such a session if the
  quorum requirements are not met, then the matter shall be referred to
  the General Body under the provision of Article 2.15 of the
  Constitution.
\item
  All Senators and Executives must, as far as possible, attend all
  sessions of the Students' Senate. However, under special
  circumstances, they shall be allowed to miss sessions of the Students'
  Senate with the permission of the Chairperson. The rules pertaining to
  the attendance of Senators and Executives in the sessions of the
  Students' Senate shall be:

  \begin{enumerate}
  \def\labelenumii{\alph{enumii}.}
  \item
    A Senator shall not miss more than 1 session without permission.The
    Chairperson shall issue a show-cause notice to a Senator after
    he/she has missed 2 sessions without permission. He will be further
    removed from the post if the Students' Senate is not satisfied with
    his defence or if he misses another session
  \item
    A Senator shall not miss more than 3 sessions with or without
    permission. The Chairperson shall issue a show-cause notice to a
    Senator after he/she has missed 4 sessions with or without
    permission. He will be further removed from the post if the
    Students' Senate is not satisfied with his defence or if he misses
    another session
  \item
    An Executive may not be absent from a session without prior
    permission. If an Executive is absent from a session without prior
    permission, the Chairperson shall issue a show-cause notice to the
    Executive. If two-thirds of the Students' Senate is unsatisfied with
    his/her response, then the Executive shall stand removed from the
    post.
  \item
    If an Executive is absent from a session without prior permission
    for a second time, he/she shall stand removed from the post
  \item
    A Senator may send a nominee to 20\% of the sessions that have as
    yet occurred rounded down or 2 sessions, whichever is higher. The
    nominee must be of the same batch as the Senator.
  \item
    The Emergency sessions shall not count towards the total number of
    sessions of the Students' Senate. However, the sessions adjourned
    due to a lack of quorum shall count.
  \item
    Written permission is to be sought for absence from a session and
    for sending a nominee to a session . This permission should be
    obtained before a deadline which shall be fixed by the Chairperson,
    who shall respond as soon as possible.
  \item
    Members shall not leave during sessions without the permission of
    the Steering Committee of the Students' Senate. If a Senator leaves
    the session for more than 20 minutes without permission, he shall be
    considered absent without permission.
  \end{enumerate}
\item
  All members of the Students' Senate, General Body Members and special
  invitees who are present in a session of the Students' Senate shall
  follow the Code of Conduct, which is as follows:

  \begin{enumerate}
  \def\labelenumii{\alph{enumii}.}
  \item
    Members of the Students' Senate, Members of the General Body and
    Special Invitees shall have to take the permission of the
    Chairperson before expressing a viewpoint.
  \item
    Members of the Students' Senate, Members of the General Body and
    Special Invitees speaking on the floor of the house shall address
    the Chairperson during the course of discussion.
  \item
    Members of the Students' Senate, Members of the General Body and
    Special Invitees should not make any irrelevant personal comment on
    the floor of the house or make any coarse remarks of an offensive
    nature.
  \end{enumerate}
\item
  The sessions of the Students' Senate shall ordinarily be open to the
  General Body of the Students. Members of the General body may express
  their views either through the members of the Students' Senate or ask
  for permission to speak themselves. However, the Students' Senate may
  resolve to hold a closed door session on the basis of a simple
  majority. The Chairperson shall request all members of the general
  body of the Students to leave in case any confidential matter comes up
  for discussion. The Chairperson may request a General Body member to
  leave if found violating the Code of Conduct as defined in Article
  A.1.12
\item
  The Students' Students' Senate may by consensus invite the Patron,
  Students' Gymkhana , Counsellors of the Students' Gymkhana or other
  Special Invitees as and when thought necessary.
\item
  The responsibilities for the recording of the minutes of the sessions
  of the Students' Senate shall rest with the Chairperson. The actual
  recording of the minutes shall be done by the Steering Committee of
  the Students' Senate. The recorded minutes shall constitute a draft
  and not an official record.
\item
  The draft minutes shall be published and circulated by the Chairperson
  amongst:

  \begin{enumerate}
  \def\labelenumii{\alph{enumii}.}
  \item
    Members of the Students' Senate
  \item
    The General Body of the Students.
  \end{enumerate}

  Within a period of two weeks of the session in question, if this has
  not been done, all subsequent sessions of the students Students'
  Senate shall be compulsorily adjourned if so demanded by even one
  member of the Students' Senate until such time as the minutes are
  circulated.
\item
  When the draft minutes have been circulated they shall be placed for
  confirmation at the first subsequent non-emergency session of the
  Students' Senate. However, if less than twenty-four hours have elapsed
  since the circulation of the draft minutes, confirmation shall be done
  in the next session if so demanded by even one member of the Students'
  Senate.
\item
  Any change in the minutes during confirmation may not be confirmed in
  the same session, and must be brought again in the next session of the
  Students' Senate. However, the other parts of the minutes may be
  partially confirmed by the Students' Senate. The changes shall be
  included in full in the minutes of the session in which confirmation
  of the changes is done.
\item
  On confirmation of the minutes, the Chairperson shall sign the
  ratified minutes, whereupon they shall become official.
\end{enumerate}

\subsection{Part 2: Sub-Committees of the Students'
Senate}\label{part-2-sub-committees-of-the-students-senate}

\begin{enumerate}
\def\labelenumi{\arabic{enumi}.}
\item
  General Provisions:

  \begin{enumerate}
  \def\labelenumii{\alph{enumii}.}
  \item
    The Students' Senate may, upon consideration of matters put before
    it, decide to, by majority, refer them to an appointed subcommittee
    of the Students' Senate if said matters require deeper thought and
    investigation than can be given during a session of the Students'
    Senate. These sub-committees may be either ad-hoc or standing.
  \item
    Unless otherwise explicitly specified in the terms of reference:
  \end{enumerate}

  \begin{enumerate}
  \def\labelenumii{\arabic{enumii}.}
  \item
    No sub-committee of the Students' Senate shall have any executive
    function whatsoever.
  \item
    The sub-committee shall come into existence immediately upon the
    appointment of its members.
  \end{enumerate}

  \begin{enumerate}
  \def\labelenumii{\alph{enumii}.}
  \setcounter{enumii}{2}
  \item
    All ad-hoc sub-committees of the Students' Senate shall go out of
    existence when the outgoing Students' Senate hands over charge to
    the incoming Students' Senate. The incoming Students' Senate in its
    1st session may choose to reconstitute all the committees that went
    out of existence with new members if necessary.
  \item
    The standing subcommittees will continue to function till charge is
    handed over to the incoming committees.
  \item
    An ad-hoc sub-committee of the Students' Senate may be converted
    into a standing committee by the Students' Senate. This shall be
    taken up as a constitutional amendment.
  \end{enumerate}
\item
  Formation:

  \begin{enumerate}
  \def\labelenumii{\alph{enumii}.}
  \itemsep1pt\parskip0pt\parsep0pt
  \item
    Prior to the formation of any sub-committee, the Students' Senate
    shall decide by a simple majority the following:
  \end{enumerate}

  \begin{enumerate}
  \def\labelenumii{\arabic{enumii}.}
  \item
    Its terms of reference
  \item
    Its strength
  \item
    In the case of an ad-hoc sub-committee, its period of existence
  \end{enumerate}

  \begin{enumerate}
  \def\labelenumii{\alph{enumii}.}
  \setcounter{enumii}{1}
  \itemsep1pt\parskip0pt\parsep0pt
  \item
    The procedure for appointing of members to ad - hoc committees shall
    be as follows:
  \end{enumerate}

  \begin{enumerate}
  \def\labelenumii{\arabic{enumii}.}
  \item
    The Chairperson of the Students' Senate shall invite nominations for
    membership of an ad-hoc sub-committee under formation, from
    senators.
  \item
    Senators and General Body Members present shall nominate themselves
    or any other General Body Member with experience relevant to the
    working of the sub- committee.
  \item
    If the number of nominations received is in excess of the proposed
    strength, the Senate shall select the members of the sub-committee
    from the nominations received.
  \item
    If the number of nominations received is less than the proposed
    strength of the ad- hoc sub-committee, the Chairperson shall make a
    second call for the same. If the number is in excess of the proposed
    strength, the procedure laid down in Article A.2.4.b iii) shall be
    followed.
  \item
    In case insufficient nominations are received even after the second
    call, the Chairperson shall nominate members to the remaining posts
    of the sub-committee at his own discretion. If no nominations are
    received the Chairperson shall officiate as the Convener of that
    sub- committee and shall appoint one UG and one PG member to
    discharge the functions of that sub-committee. The sub- committee
    thus formed, must be ratified by the Students' Senate.
  \item
    The Chairperson may include any number of special invitees, over and
    above the strength of the committee as approved by the Students'
    Senate, into the sub-committee on the basis of their experience or
    special knowledge.
  \end{enumerate}

  \begin{enumerate}
  \def\labelenumii{\alph{enumii}.}
  \setcounter{enumii}{2}
  \item
    The Students' Senate shall elect the Convener of the ad-hoc sub-
    committee from amongst the members of the sub-committee.
  \item
    In case any post of a sub-committee of the Students' Senate falls
    vacant in the middle of its term, the Chairperson will call for
    nominations for that post. If the post of the Convener of an ad-hoc
    sub-committee falls vacant then after electing a new member to the
    sub-committee the Students' Senate shall elect a new the Convener of
    the sub-committee.
  \end{enumerate}
\item
  Duties of Convener

  \begin{enumerate}
  \def\labelenumii{\alph{enumii}.}
  \item
    The Convener of a sub-committee may ask for a preliminary discussion
    on the floor of the Students' Senate if he/she thinks that the same
    is necessary.
  \item
    The Convener of a sub-committee shall be responsible for the proper
    functioning of the sub-committee. He/she shall:
  \end{enumerate}

  \begin{enumerate}
  \def\labelenumii{\arabic{enumii}.}
  \item
    Convene and preside over all sessions of that sub-committee.
  \item
    Coordinate its activities and ensure its compliance with the terms
    of reference.
  \item
    Be directly responsible for all communication within the
    sub-committee. He/she shall route all business with other
    individuals or bodies through the Chairperson, who shall be obliged
    to render the assistance asked for.
  \item
    Ensure that an adequate opportunity has been provided to the general
    body of the students to express their views.
  \item
    Be responsible for the preparation of the sub-committee's report to
    the Students' Senate within the time specified in terms of
    reference.
  \item
    Submit the final report of the sub-committee to the Chairperson, who
    shall, circulate the entire text of the report as a part of the
    agenda for a subsequent session of the Students' Senate.
  \end{enumerate}
\item
  Sessions

  \begin{enumerate}
  \def\labelenumii{\alph{enumii}.}
  \item
    Ordinarily, all recommendations of a sub-committee shall be by
    consensus. In case of dissent, the dissenting members may attach a
    note of dissent to the report of the sub-committee for the
    consideration of the Students' Senate
  \item
    All sessions of the ad hoc sub - committees and standing committees
    of the Students' Senate shall require quorum, which is constituted
    by one half of the strength of the committee rounded up to the
    nearest integer. All members of the committees who are members of
    the Students' Senate or are nominated by the Students' Senate, are
    required to be present in at least half the sessions held until then
    rounded down to the nearest integer. If any member falls short of
    this attendance requirement, the Convener of the committee may
    report this to the Chairperson of the Students' Senate who may call
    for either a written explanation or a verbal explanation in the
    Senate
  \end{enumerate}
\end{enumerate}

\subsection{3: Office and Standing Committees of the Students'
Senate}\label{office-and-standing-committees-of-the-students-senate}

\begin{enumerate}
\def\labelenumi{\arabic{enumi}.}
\item
  Office of the Students' Senate

  \begin{enumerate}
  \def\labelenumii{\alph{enumii}.}
  \itemsep1pt\parskip0pt\parsep0pt
  \item
    An office of the Students' Senate shall be maintained by the
    Chairperson of the Students' Senate, who shall be responsible for
    the up to date maintenance of the following:
  \end{enumerate}

  \begin{enumerate}
  \def\labelenumii{\arabic{enumii}.}
  \item
    The Constitution of the Students' Gymkhana with amendments.
  \item
    Minutes of the sessions of the Students' Senate.
  \item
    The Rules and Procedures of the Students' Senate with amendments.
  \item
    Reports of the sub-committees of the Students' Senate.
  \item
    All correspondence relating to the convening and functioning of the
    Students' Senate.
  \end{enumerate}

  \begin{enumerate}
  \def\labelenumii{\alph{enumii}.}
  \setcounter{enumii}{1}
  \itemsep1pt\parskip0pt\parsep0pt
  \item
    The Chairperson shall be responsible for the production, on demand,
    of all the above documents in original to the following:
  \end{enumerate}

  \begin{enumerate}
  \def\labelenumii{\arabic{enumii}.}
  \item
    Members of the Students' Senate.
  \item
    Members of the general Body of the Students
  \item
    Any other person or body authorised by the Students' Senate.
  \end{enumerate}

  \begin{enumerate}
  \def\labelenumii{\alph{enumii}.}
  \setcounter{enumii}{2}
  \itemsep1pt\parskip0pt\parsep0pt
  \item
    To assist the Chairperson in the discharge of his/her
    responsibilities, as mentioned above or otherwise, he/she may,
    through a call for nominations to the general body, nominate
    Secretary(s) under the Office of Students' Senate, the number and
    responsibilities of whom may be ascertained by the Senate from time
    to time based on the recommendation of the Chairperson, Students'
    Senate. The name(s) of the person(s) for the post(s) shall be
    proposed by the Chairperson to the Senate for ratification.
  \end{enumerate}
\item
  The Steering committee:

  \begin{enumerate}
  \def\labelenumii{\alph{enumii}.}
  \item
    The Steering committee shall be responsible for the optimum
    utilization of the sessions of the Students' Senate. It shall
    prepare the agenda for every session of the Students' Senate.
  \item
    The Steering committee shall also be responsible for the recording
    of the minutes of the sessions of the Students' Senate, preparing
    the draft minutes and maintaining the attendance records
  \item
    The Steering committee shall also provide any other assistance
    required by:
  \end{enumerate}

  \begin{enumerate}
  \def\labelenumii{\arabic{enumii}.}
  \item
    The Chairperson, of the Students' Senate
  \item
    The Students' Senate
  \item
    The sub-committees of the Students' Senate
  \end{enumerate}
\item
  The Rules and Procedures Committee:
\end{enumerate}

\begin{enumerate}
\def\labelenumi{\alph{enumi}.}
\itemsep1pt\parskip0pt\parsep0pt
\item
  The Rules and Procedures committee shall be responsible for the
  observance of due procedure in the Students' Senate. It shall:
\end{enumerate}

\begin{enumerate}
\def\labelenumi{\arabic{enumi}.}
\item
  Prepare and propose changes to the Constitution and its Appendices

  \begin{enumerate}
  \def\labelenumii{\arabic{enumii}.}
  \setcounter{enumii}{1}
  \item
    Advise the Students' Senate on all matters concerning the
    Constitution and its Appendices.
  \item
    Study all proposed changes in the Constitution and its Appendices
    and give its opinion on the same.
  \item
    Advise the Students' Senate on all matters relating to the formation
    and functioning of its sub-committees.
  \end{enumerate}
\end{enumerate}

\begin{enumerate}
\def\labelenumi{\alph{enumi}.}
\setcounter{enumi}{1}
\itemsep1pt\parskip0pt\parsep0pt
\item
  All recommendations of the rules and procedures committee shall be
  subject to the approval by a simple majority of the Students' Senate.
\end{enumerate}

\begin{enumerate}
\def\labelenumi{\arabic{enumi}.}
\setcounter{enumi}{3}
\item
  Nominations Committee

  \begin{enumerate}
  \def\labelenumii{\alph{enumii}.}
  \itemsep1pt\parskip0pt\parsep0pt
  \item
    The Nominations Committee shall recommend the names of the following
    to the Senate:
  \end{enumerate}

  \begin{enumerate}
  \def\labelenumii{\arabic{enumii}.}
  \item
    Students' Senate Representative(s) to the Standing Committees of the
    (Academic) Senate.
  \item
    Students' Senate Representative(s) to the various Standing
    Committees of Departments/Interdisciplinary Programmes.
  \item
    Core Team of the Cells
  \end{enumerate}

  \begin{enumerate}
  \def\labelenumii{\alph{enumii}.}
  \setcounter{enumii}{1}
  \item
    The Nominations Committee shall also recommend names of Student's
    Senate Representative(s) to all institute committees in which they
    are required.
  \item
    The Nominations Committee can call for an explanation, in case its
    recommended nominee(s) is (are) not found suitable for the assigned
    task. All its nominations/actions are to be ratified by the Senate.
  \item
    For each selection pertaining to any representation/position, the
    Nominations Committee may constitute a panel to interview the
    various candidates and select from them the names that shall be
    recommended to the Senate.
  \end{enumerate}
\item
  Committee of Festival Affairs (COFA)

  \begin{enumerate}
  \def\labelenumii{\alph{enumii}.}
  \item
    The COFA and the festivals shall be governed by the Festival Manual,
    which shall be an Appendix to the Constitution.
  \item
    The ex-officio Chairpersons of the respective COFAs and shall
    maintain the records of their respective COFA. He/she shall
    represent the COFA in the sessions of the Senate, and the Senate in
    the meetings of the COFA.
  \item
    Joint meetings of the COFA shall be called by the President, either
    on his own initiative or when so requisitioned by the Chairperson to
    discuss matters that affect all festivals. Such meetings shall be
    presided over by the President. The recommendations of the COFA in
    such a meeting shall be forwarded to the Senate for approval.
  \end{enumerate}
\item
  Finance Committee

  \begin{enumerate}
  \def\labelenumii{\alph{enumii}.}
  \item
    The Finance Committee shall report to the Senate the Finances and
    Accounts of the Gymkhana periodically.
  \item
    Each member of the Finance Committee shall be associated (at the
    recommendation of the Chairperson, Students' Senate) with a Gymkhana
    festival, and shall be responsible to the Finance Committee Senate
    for keeping the finances of the festival in check. He/she shall be
    an ex-officio member of the corresponding COFA, and may not be a
    member of any festival conduction team.
  \end{enumerate}
\end{enumerate}

\subsection{Part 4: Procedures}\label{part-4-procedures}

\begin{enumerate}
\def\labelenumi{\arabic{enumi}.}
\item
  Changes, accepted by the Students' Senate, to the Constitution shall
  be incorporated as amendments to the Constitution. The procedure for
  amending the Constitution shall be as follows:

  \begin{enumerate}
  \def\labelenumii{\alph{enumii}.}
  \item
    A constitutional amendment shall require a two-third majority for
    its passage in the Students' Senate.
  \item
    The constitutional amendment may be discussed in any session of the
    Students' Senate provided the following have been included in the
    previously circulated agenda for the same.
  \item
    Written notice of the proposed amendment, giving the text of the
    proposed amendment together with a statement of its purpose and
    effect.
  \item
    The opinion of the Rules and Procedures Committee on the proposed
    amendment.
  \item
    The voting on a constitutional amendment shall take place in the
    non- emergency session succeeding the one in which the discussion on
    the same was closed. Reopening of the discussion prior to voting is
    permitted.
  \end{enumerate}
\item
  Change in the Appendices of the Constitution:

  \begin{enumerate}
  \def\labelenumii{\alph{enumii}.}
  \item
    Inadequacies in the Appendices of the Constitution shall be referred
    by the Chairperson of the Students' Senate to the Convener of the
    Rules and Procedures Committee for corrections.
  \item
    Amendments to the Appendices of the Constitution shall require a
    simple majority in the Senate for passing.
  \end{enumerate}
\item
  A member may raise a Point of Order, if in the opinion of the member,

  \begin{enumerate}
  \def\labelenumii{\alph{enumii}.}
  \item
    Any constitutional provision or rule or procedure is being
    transgressed.
  \item
    Any established convention of the Students' Senate is being
    transgressed.
  \item
    An objectionable procedure is being followed.
  \item
    If any member wishes to raise a Point of Order, the Chairperson must
    admit the point immediately, with greater precedence than any other
    point or motion placed before the house
  \end{enumerate}
\item
  A decision of the Students' Senate may be arrived at by consensus if
  the proposal:

  \begin{enumerate}
  \def\labelenumii{\alph{enumii}.}
  \item
    Does not require a majority vote for its passage under any of the
    rules and procedures.
  \item
    Is not objected to by any member of the Students' Senate or if all
    objections are withdrawn.
  \item
    Is not meant to give expression to the opinion of the general body
    of the students.
  \end{enumerate}
\item
  In case a decision of the Students' Senate is meant to give expression
  to the opinion of the general body of the students, the Students'
  Senate shall decide by resolution. The procedure for adopting a
  resolution is as follows:

  \begin{enumerate}
  \def\labelenumii{\alph{enumii}.}
  \item
    All resolutions shall be submitted in writing by the proposer to the
    Chairperson
  \item
    All resolutions shall be seconded by at least one member of the
    Students' Senate other than the proposer.
  \item
    A resolution shall be considered tabled only when it has been read
    out by the Chairperson.
  \item
    A resolution can be tabled only if both the proposer and seconder
    are present in the session.
  \item
    No prior notice is necessary for the tabling of a resolution unless
    otherwise required by any of these rules and procedures.
  \item
    While discussion on a resolution is in progress amendments to the
    resolution may be moved. Incorporation of the proposed amendments
    is, however, subject to the discretion of the proposer and seconders
    of the resolution.
  \item
    If the proposed amendment is not accepted by the proposer of the
    resolution, the proposer of the amendment may propose an alternate
    resolution incorporating his/her proposed amendment for simultaneous
    consideration.
  \item
    While discussion on a resolution is in progress, further resolution
    concerning the same matter may be tabled for simultaneous
    consideration of the Students' Students' Senate, but no resolution
    concerning any other matter shall be tabled. When all the
    resolutions on the table have been adequately discussed, the
    Chairperson shall declare the discussion closed. The Chairperson
    shall then read out all the resolutions in their final from in the
    order in which they were tabled prior to voting.
  \item
    Once the discussion on a resolution has been closed, it shall be
    reopened only with the permission of the Chairperson
  \item
    Resolutions shall be taken up one at a time for voting in the order
    in which they were tabled.
  \item
    A secret ballot, if so requested by even one member of the Students'
    Senate, shall be permissible.
  \end{enumerate}
\item
  Calling Attention Motion:

  \begin{enumerate}
  \def\labelenumii{\alph{enumii}.}
  \itemsep1pt\parskip0pt\parsep0pt
  \item
    A member of the Students' Senate may move a Calling Attention Motion
    either verbally or in writing to:
  \end{enumerate}

  \begin{enumerate}
  \def\labelenumii{\arabic{enumii}.}
  \item
    Attract the attention of the Students' Senate or any Executive of
    the Students' Gymkhana or the General Body of the Students to say
    particular matter, or
  \item
    Address any question to the Students' Senate or any executive of the
    Students' Gymkhana, in which case the member may ask a written
    answer.
  \end{enumerate}

  \begin{enumerate}
  \def\labelenumii{\alph{enumii}.}
  \setcounter{enumii}{1}
  \itemsep1pt\parskip0pt\parsep0pt
  \item
    In case a calling attention motion is in the form of a question, the
    individual or body to which the question is addressed may ask for:
  \end{enumerate}

  \begin{enumerate}
  \def\labelenumii{\arabic{enumii}.}
  \item
    The question to be submitted in writing.
  \item
    Any reasonable amount of time for preparing the answer.
  \end{enumerate}

  \begin{enumerate}
  \def\labelenumii{\alph{enumii}.}
  \setcounter{enumii}{2}
  \itemsep1pt\parskip0pt\parsep0pt
  \item
    The answer to a calling attention motion shall be recorded in full
    in the minutes.
  \end{enumerate}
\item
  Adjournment Motions:

  \begin{enumerate}
  \def\labelenumii{\alph{enumii}.}
  \item
    An Adjournment Motion may be moved by a member of the Students'
    Senate while a session is in progress, if and when he/she thinks the
    same is desirable.
  \item
    If the adjournment asked for exceeds thirty minutes the Adjournment
    Motion shall be in writing
  \item
    The Chairperson shall put an Adjournment Motion to vote as soon as
    possible after it has been proposed.
  \item
    Adjournment motions shall be subject to a simple majority in the
    Students' Senate.
  \item
    The Chairperson may, if necessary, adjourn the session for at most
    ten minutes without the necessity of seeking the Students' Senate's
    the vote. This should be exercised only in case of extreme disorder
  \end{enumerate}
\item
  Caucus Motion:

  \begin{enumerate}
  \def\labelenumii{\alph{enumii}.}
  \item
    A member of the Students' Senate may move a motion for caucus either
    verbally or in writing if the discussion or debate on the Students'
    Senate floor goes on for too long due to lack of consensus.
  \item
    If 1/3rd of the present and voting Senators vote in favour of the
    motion, the Students' Senate shall enter into an unmoderated Caucus
    for 5-10 minutes wherein Senators shall mingle among each other to
    freely discuss their opinion on the agenda item and resolve issues
    individually
  \item
    With the consent of half the present and voting Senators, a Caucus
    may be extended up to 20 minutes. No Caucus may last for more than
    20 minutes.
  \item
    A motion for caucus shall be called only once for any agenda item.
  \end{enumerate}
\item
  Any position holder who derives his or her authority from the
  Students' Gymkhana can be reprimanded for misconduct or dereliction of
  duty through a Censure Motion:

  a.All Censure Motions shall be submitted in writing to the
  Chairperson.

  \begin{enumerate}
  \def\labelenumii{\alph{enumii}.}
  \setcounter{enumii}{1}
  \item
    If the Chairperson is the defendant, he/she shall vacate the chair
    and the Parliamentarian of the Students' Senate shall chair the
    session.
  \item
    All Censure Motions shall be duly proposed and seconded by at least
    two members of the Students' Senate other than the proposer.
  \item
    A Censure Motion may be proposed and discussed in the absence of the
    defendant.
  \item
    No Censure Motion shall be put to vote to unless;
  \end{enumerate}

  \begin{enumerate}
  \def\labelenumii{\arabic{enumii}.}
  \item
    The defendant has been given an opportunity to the satisfaction of
    the defendant to defend himself/herself before the Students' Senate
    and
  \item
    Adequate discussion has followed the defendant's defence.
  \end{enumerate}

  \begin{enumerate}
  \def\labelenumii{\alph{enumii}.}
  \setcounter{enumii}{5}
  \item
    The defendant shall be asked to leave the session by the Chairperson
    when the Censure Motion is put to vote. The defendant shall not be
    entitled to vote on the Censure Motion.
  \item
    Voting on a Censure Motion shall be by secret ballot if so demanded
    by even one member of the Students' Senate.
  \item
    All Censure Motions shall be subject to a simple majority in the
    Students' Senate.
  \item
    Notice of the Censure motion, once passed, shall be sent out to the
    General Body along with the details of the individual censured,
    his/her post and transgressions. Further, depending on the severity
    of the transgressions, after passing the Censure Motion, the
    Students' Senate may take one or more steps of those given below:
  \end{enumerate}

  \begin{enumerate}
  \def\labelenumii{\arabic{enumii}.}
  \item
    Direct the student to mention the details of the Censure Motion
    along with the achievements and work done as a part of holding that
    position. This shall be informed to the Placement Office and other
    authorities as well to ensure that it is implemented.
  \item
    The details about the Censure Motion shall be submitted to the
    Office of Dean of Academic Affairs to be added to the individual's
    records as maintained by the institute.
  \end{enumerate}
\item
  No-Confidence Motions:

  \begin{enumerate}
  \def\labelenumii{\alph{enumii}.}
  \item
    All No-Confidence Motions shall be submitted in writing to the
    Chairperson.
  \item
    If the Chairperson is the defendant, he/she shall vacate the chair
    and the Parliamentarian of the Students' Senate shall chair the
    session.
  \item
    All No-Confidence Motions shall be duly proposed and seconded by at
    least two members of the Students' Senate other than the proposer.
  \item
    A No-Confidence Motion shall be discussed only in the presence of
    the defendant.
  \item
    No-Confidence Motion shall not be put to vote unless:
  \end{enumerate}

  \begin{enumerate}
  \def\labelenumii{\arabic{enumii}.}
  \itemsep1pt\parskip0pt\parsep0pt
  \item
    The Defendant has been given an opportunity to the satisfaction of
    the defendant to defend himself/herself before the Students' Senate
  \item
    Adequate discussion has followed the Defendant's defence
  \end{enumerate}

  \begin{enumerate}
  \def\labelenumii{\alph{enumii}.}
  \setcounter{enumii}{5}
  \item
    In case of repeated absence the Students' Senate may consider the
    case in the absence of the Defendant.
  \item
    The defendant shall be asked to leave the session by the Chairperson
    when the No- Confidence Motion is put to vote.
  \item
    Voting on a No-Confidence Motion shall be by secret ballot.
  \item
    A No-Confidence Motion against an executive shall require a 2/3rd
    majority to be passed while a No- Confidence Motion against all
    other office bearers shall require a simple majority.
  \end{enumerate}

  In case a No-Confidence Motion is passed, the Defendant shall stand
  unseated from the office on the declaration of the results of the
  voting.
\end{enumerate}
;